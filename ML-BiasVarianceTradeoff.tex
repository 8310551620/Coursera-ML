
\documentclass[11pt]{article} % use larger type; default would be 10pt
\usepackage{framed}
\usepackage[utf8]{inputenc} % set input encoding (not needed with XeLaTeX)
\usepackage{geometry} % to change the page dimensions
\geometry{a4paper} % or letterpaper (US) or a5paper or....
% \geometry{margin=2in} % for example, change the margins to 2 inches all round
% \geometry{landscape} % set up the page for landscape
%   read geometry.pdf for detailed page layout information

\usepackage{graphicx} % support the \includegraphics command and options

% \usepackage[parfill]{parskip} % Activate to begin paragraphs with an empty line rather than an indent

%%% PACKAGES
\usepackage{booktabs} % for much better looking tables
\usepackage{array} % for better arrays (eg matrices) in maths
\usepackage{paralist} % very flexible & customisable lists (eg. enumerate/itemize, etc.)
\usepackage{verbatim} % adds environment for commenting out blocks of text & for better verbatim
\usepackage{subfig} % make it possible to include more than one captioned figure/table in a single float
% These packages are all incorporated in the memoir class to one degree or another...

%%% HEADERS & FOOTERS
\usepackage{fancyhdr} % This should be set AFTER setting up the page geometry
\pagestyle{fancy} % options: empty , plain , fancy
\renewcommand{\headrulewidth}{0pt} % customise the layout...
\lhead{}\chead{}\rhead{}
\lfoot{}\cfoot{\thepage}\rfoot{}

%%% SECTION TITLE APPEARANCE
\usepackage{sectsty}
\allsectionsfont{\sffamily\mdseries\upshape} % (See the fntguide.pdf for font help)
% (This matches ConTeXt defaults)

%%% ToC (table of contents) APPEARANCE
\usepackage[nottoc,notlof,notlot]{tocbibind} % Put the bibliography in the ToC
\usepackage[titles,subfigure]{tocloft} % Alter the style of the Table of Contents
\renewcommand{\cftsecfont}{\rmfamily\mdseries\upshape}
\renewcommand{\cftsecpagefont}{\rmfamily\mdseries\upshape} % No bold!


\title{Computing for Data Analysis}
\author{Dublin \texttt{R}}

\begin{document}
\section{Understanding the Bias-Variance Tradeoff}
Source:  http://scott.fortmann-roe.com/docs/BiasVariance.html
\begin{itemize}
\item When we discuss prediction models, prediction errors can be decomposed into two main subcomponents we care about: error due to ``bias" and error due to ``variance".  \item There is a tradeoff between a model's ability to minimize bias and variance. Understanding these two types of error can help us diagnose model results and avoid the mistake of over- or under-fitting.

\item Bias and Variance
Understanding how different sources of error lead to bias and variance helps us improve the data fitting process resulting in more accurate models. We define bias and variance in three ways: conceptually, graphically and mathematically.

\item Conceptual Definition
Error due to Bias: The error due to bias is taken as the difference between the expected (or average) prediction of our model and the correct value which we are trying to predict. Of course you only have one model so talking about expected or average prediction values might seem a little strange. 
\item However, imagine you could repeat the whole model building process more than once: each time you gather new data and run a new analysis creating a new model. Due to randomness in the underlying data sets, the resulting models will have a range of predictions. Bias measures how far off in general these models' predictions are from the correct value.
\item Error due to Variance: The error due to variance is taken as the variability of a model prediction for a given data point. Again, imagine you can repeat the entire model building process multiple times. The variance is how much the predictions for a given point vary between different realizations of the model.

\end{itemize}

\end{document}