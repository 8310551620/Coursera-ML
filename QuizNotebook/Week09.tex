Close
Recommender Systems

5 questions

%=====================================================================%

1. 
Suppose you run a bookstore, and have ratings (1 to 5 stars)

of books. Your collaborative filtering algorithm has learned

a parameter vector \theta(j) for user j, and a feature

vector $ \times $(i) for each book. You would like to compute the

"training error", meaning the average squared error of your

system's predictions on all the ratings that you have gotten

from your users. Which of these are correct ways of doing so (check all that apply)?

For this problem, let m be the total number of ratings you

have gotten from your users. (Another way of saying this is

that m=∑nmi=1∑nuj=1r(i,j)). [Hint: Two of the four options below are correct.]


1m∑(i,j):r(i,j)=1((\theta(j))T$ \times $(i)−r(i,j))2 SELECTED

1m∑(i,j):r(i,j)=1((\theta(j))T$ \times $(i)−y(i,j))2

1m∑nmi=1∑j:r(i,j)=1(∑nk=1(\theta(j))k$ \times $(i)k−y(i,j))2 SELECTED

1m∑nuj=1∑i:r(i,j)=1(∑nk=1(\theta(k))j$ \times $(k)i−y(i,j))2

%=====================================================================%

2. 

In which of the following situations will a collaborative filtering system be the most appropriate learning algorithm (compared to linear or logistic regression)?

\begin{itemize}
\item WRONG You're an artist and hand-paint portraits for your clients. Each client gets a different portrait (of themselves) and gives you 1-5 star rating feedback, and each client purchases at most 1 portrait. You'd like to predict what rating your ne$ \times $t customer will give you.

\item SELECTED You manage an online bookstore and you have the book ratings from many users. You want to learn to predict the e$ \times $pected sales volume (number of books sold) as a function of the average rating of a book.

\item SELECTED You own a clothing store that sells many styles and brands of jeans. You have collected reviews of the different styles and brands from frequent shoppers, and you want to use these reviews to offer those shoppers discounts on the jeans you think they are most likely to purchase

\item WRONG You run an online bookstore and collect the ratings of many users. You want to use this to identify what books are "similar" to each other (i.e., if one user likes a certain book, what are other books that she might also like?)
\end{itemize}
%=====================================================================%

3. 

You run a movie empire, and want to build a movie recommendation system based on collaborative filtering. There were three popular review websites (which we'll call A, B and C) which users to go to rate movies, and you have just acquired all three companies that run these websites. You'd like to merge the three companies' datasets together to build a single/unified system. On website A, users rank a movie as having 1 through 5 stars. On website B, users rank on a scale of 1 - 10, and decimal values (e.g., 7.5) are allowed. On website C, the ratings are from 1 to 100. You also have enough information to identify users/movies on one website with users/movies on a different website. Which of the following statements is true?

\begin{itemize}
\item You can combine all three training sets into one without any modification and e$ \times $pect high performance from a recommendation system.

\item It is not possible to combine these websites' data. You must build three separate recommendation systems.

\item CORRECT You can merge the three datasets into one, but you should first normalize each dataset separately by subtracting the mean and then dividing by (ma$ \times $ - min) where the ma$ \times $ and min (5-1) or (10-1) or (100-1) for the three websites respectively.

\item You can combine all three training sets into one as long as your perform mean normalization and feature scaling after you merge the data.

\end{itemize}

%=====================================================================%
4. 
Which of the following are true of collaborative filtering systems? Check all that apply.

\begin{itemize}

\item WRONG Suppose you are writing a recommender system to predict a user's book preferences. In order to build such a system, you need that user to rate all the other books in your training set.

\item CORRECT For collaborative filtering, it is possible to use one of the advanced optimization algoirthms (L-BFGS/conjugate gradient/etc.) to solve for both the $ \times $(i)'s and \theta(j)'s simultaneously.

\item CORRECT Even if each user has rated only a small fraction of all of your products (so r(i,j)=0 for the vast majority of (i,j) pairs), you can still build a recommender system by using collaborative filtering.

\item WRONG For collaborative filtering, the optimization algorithm you should use is gradient descent. In particular, you cannot use more advanced optimization algorithms (L-BFGS/conjugate gradient/etc.) for collaborative filtering, since you have to solve for both the $ \times $(i)'s and \theta(j)'s simultaneously.

\end{itemize}

%=====================================================================%

5. 

Suppose you have two matrices A and B, where A is 5$ \times $3 and B is 3$ \times $5. Their product is C=AB, a 5$ \times $5 matri$ \times $. Furthermore, you have a 5$ \times $5 matri$ \times $ R where every entry is 0 or 1. You want to find the sum of all elements C(i,j) for which the corresponding R(i,j) is 1, and ignore all elements C(i,j) where R(i,j)=0. One way to do so is the following code:


Which of the following pieces of Octave code will also correctly compute this total? Check all that apply. Assume all options are in code.

\begin{itemize}

\item CORRECT total = sum(sum((A * B) .* R))

\item CORRECT C = (A * B) .* R; total = sum(C(:));

\item WRONG total = sum(sum((A * B) * R));

\item WRONG C = (A * B) * R; total = sum(C(:));
\end{itemize}
%=====================================================================%

