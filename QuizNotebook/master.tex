You'd like to use polynomial regression to predict a student's ჌▘nal exam score
from their midterm exam score. Concretely, suppose you want to ჌▘t a model
of the form , where is the midterm score and
is (midterm score) . Further, you plan to use both feature scaling (dividing
by the "max-min", or range, of a feature) and mean normalization.
What is the normalized feature ? (Hint: midterm = 72, fial = 74 is training
example 2.) 


Please round your answer to two decimal places and enter in
the text box below.

%----------------------------------------%

Suppose you have training examples with features (excluding
the additional all-ones feature for the intercept term, which you should add).
The normal equation is . For the given values of and
, what are the dimensions of , , and in this equation?

%---------------------------------------%
\subsection{Question 4}
Suppose you have a dataset with m = 1000000 examples and n=200000
features for each example. You want to use multivariate linear regression to
fit the parameters $\theta$ to our data. Should you prefer gradient descent or the
normal equation?


%==========================================================================%

\section{Week 4 Neural Networks: Representation}


Topics: Neural Networks: Representation

Programming Exercise 3 : Multi-class classification and neural networks.

%=================================================================%

\section{ML Week 5}
\subsection*{Overfitting and Regularization}

\begin{itemize}
\item If one neural network overfits the training set, one reasonable step is to
increase the regularization parameter $\lambda$.
\item For computational efficiency, after we performed gradient checking to verify that our back-propagation 
code is correct, we usually disable gradient checkking before using back-propagation to train
the network
\end{itemize}
%--------------------------%

\subsection*{Exercise}
Let $ J(\theta) = 3\theta^2 + 2$

Let $\theta = 1$ and $\epsilon = 0.01$

Use the formula to numerically compute an approximation to the derivative of $\theta$
at $theta = 1$

\[
\frac{J(\theta + \epsilon) - J(\theta + \epsilon)}{2\epsilon} 
= \frac{(3(1.01)^2 + 2$)-(3(0.99)^2 + 2$)}{0.002} 
= 9.003

\]
%=================================================================%
\newpage
\section{Unsupervised Learning}
%=================================================================%
\subsection*{Question 1. }
For which of the following tasks might K-means clustering be a suitable algorithm? Select all that apply.


%=================================================================%
\subsection*{Question 2.} 
Suppose we have three cluster centroids $\mu_1$=$[1 2]^T$, $\mu_2$=$[−3 0]^T$ and $\mu_3$=$[4 2]^T$. 
Furthermore, we have a training example x(i)=$[3 1]^T$. After a cluster assignment step, what will $c^{(i)}$ be?


%=================================================================%
\subsection*{Question 3.}
K-means is an iterative algorithm, and two of the following steps are repeatedly carried out in its inner-loop. Which two?


%=================================================================%
\subsection*{Question 4. }
Suppose you have an unlabeled dataset $\{x(1),\ldots,x(m)\}$. You run K-means with 50 different random

initializations, and obtain 50 different clusterings of the data. 

What is the recommended way for choosing which one of

these 50 clusterings to use?

%=================================================================%
\subsection*{Question 5. }
Which of the following statements are true? Select all that apply.


%=================================================================%
%------------------------------------------------%
\newpage
\section{ML Week 8 : Machine Learning Clustering}

Suppose you have an unlabelled set of observations$\{ x^{(1)},x^{(2)},x^{(3)},x^{(4)}, \ldots ,x^{(n)}\}$.
You run $k-$means with 50 random initializations and obtain 50 different clusterings solution of the data.
What is the recommended way of choosing which solution to use.

\subsection{The Distortion Function}
For each of the clustering solutions, the distortion function is computed as 
\[ \frac{1}{m} \sum^{m}_{i=1} \parallel x^{(i)} - \mu_c^{(i)}\parallel ^2 \]
The optimal clustering solution is the solution that minizes the distortion function.
Since a lower value for a distortion function, implies a better clustering, you should choose the clustering with 
smallest value of the distortion function.




\subsection*{Finding Closest Centroids}

In the cluste assignment phase of the algorithm, the algorithm assigns every training example $x^{(n)}$ to the closest 
centroid, given the current position of the centroids.

\[ C^{(i)} := J \mbox{ that minimizes } \| x^{} - \mu_j \| ^2 \]

\begin{itemize}
\item $C^{(i)}$ index of the centroid clostest to $x^{(i)}$
\item $\mu_j$ is the position of the $j-$th centroid.
\end{itemize}

%---------------%
\subsection{Computing Centroid Means}

\[ 
\mu_k := \frac{1}{abs(C_k)} \sum_{i \in C_k} x^{(i)}
\]
%---------------%

A good way to initialize k-mean is to select $k$ distinct examples from the training set and set the cluster centroids equal to these selected examples.

On every iteration of $k-$means, the cost function $J(C^{(1)},C^{(2)},\ldots, C^{(n)},
\mu_1,\ldots \mu_k)$

The distortion function should either stay the same or decrease, in particular it should not increase.

%---------------%
\subsection{Recommended Applications of PCA}

\begin{itemize}
\item Data Compression: reducing the dimensionof input data $x^{(i)}$, which will be used in a supervised learning algorithm
(i.e. use PCA data so that your supervised learning algorithm runs faster.

\item Data Visualization: Reduce data to 2D ( or 3D) so that it can be plotted.
\end{itemize}


\section*{ML Week 10 Large Scale Machine Learning}

\subsection*{Stochastic Gradient Checking}

\begin{itemize}
\item In each iteration of the stochastic gradient, the algorithm needs to examine/use only one training example.
\item Each iteration updates the parameters based on the cost of only one example $ Cost (\theta,(x^{(i)},y^{(i)}))$.
\item You can use numerical gradient checking to verify if your stochastic gradient descent implementation is bug-free.
\end{itemize}
