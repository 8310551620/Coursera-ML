
\documentclass[11pt]{article} % use larger type; default would be 10pt
\usepackage{framed}
\usepackage[utf8]{inputenc} % set input encoding (not needed with XeLaTeX)
\usepackage{geometry} % to change the page dimensions
\geometry{a4paper} % or letterpaper (US) or a5paper or....
% \geometry{margin=2in} % for example, change the margins to 2 inches all round
% \geometry{landscape} % set up the page for landscape
%   read geometry.pdf for detailed page layout information

\usepackage{graphicx} % support the \includegraphics command and options

% \usepackage[parfill]{parskip} % Activate to begin paragraphs with an empty line rather than an indent

%%% PACKAGES
\usepackage{booktabs} % for much better looking tables
\usepackage{array} % for better arrays (eg matrices) in maths
\usepackage{paralist} % very flexible & customisable lists (eg. enumerate/itemize, etc.)
\usepackage{verbatim} % adds environment for commenting out blocks of text & for better verbatim
\usepackage{subfig} % make it possible to include more than one captioned figure/table in a single float
% These packages are all incorporated in the memoir class to one degree or another...
\usepackage{framed}

%%% HEADERS & FOOTERS
\usepackage{fancyhdr} % This should be set AFTER setting up the page geometry
\pagestyle{fancy} % options: empty , plain , fancy
\renewcommand{\headrulewidth}{0pt} % customise the layout...
\lhead{}\chead{}\rhead{}
\lfoot{}\cfoot{\thepage}\rfoot{}

%%% SECTION TITLE APPEARANCE
\usepackage{sectsty}
\allsectionsfont{\sffamily\mdseries\upshape} % (See the fntguide.pdf for font help)
% (This matches ConTeXt defaults)

%%% ToC (table of contents) APPEARANCE
\usepackage[nottoc,notlof,notlot]{tocbibind} % Put the bibliography in the ToC
\usepackage[titles,subfigure]{tocloft} % Alter the style of the Table of Contents
\renewcommand{\cftsecfont}{\rmfamily\mdseries\upshape}
\renewcommand{\cftsecpagefont}{\rmfamily\mdseries\upshape} % No bold!
\begin{document}

%=================================================================%

\section{ML Week 5}
\subsection*{Overfitting and Regularization}

\begin{itemize}
\item If one neural network overfits the training set, one reasonable step is to
increase the regularization parameter $\lambda$.
\item For computational efficiency, after we performed gradient checking to verify that our back-propagation 
code is correct, we usually disable gradient checkking before using back-propagation to train
the network
\end{itemize}
%--------------------------%

\subsection*{Exercise}
Let $ J(\theta) = 3\theta^2 + 2$

Let $\theta = 1$ and $\epsilon = 0.01$

Use the formula to numerically compute an approximation to the derivative of $\theta$
at $theta = 1$

\[
\frac{J(\theta + \epsilon) - J(\theta + \epsilon)}{2\epsilon} 
\]
\[
= \frac{(3(1.01)^2 + 2) - (3(0.99)^2 + 2)}{0.002} 
= 9.003
\]
%=================================================================%

\end{document}